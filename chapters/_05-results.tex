\chapter{Discussion} \label{discussionchap}
\textbf{What are the main factors hindering developers in making universal designed ICT solutions?}

\section{Organisational}
\subsection{Anchoring}

\section{Process}
\subsection{Views on disability}
The participants in the workshop would be considered as having a charity view on disability, as they would only consider universal design to be relevant for their project if one of the end users had some sort of impairment. 

\begin{displayquote}
The only time I think it would be brought up at all is if there is someone struggling with it that is going to be using the solution, had that person given a feedback.
\end{displayquote}

The participants said
\begin{displayquote}
    Our app is not aimed at the public. And because of that universal design is not a requirement, it will not be until 2020 or something like that, on internal applications. So the customer will not currently focus very much on it.
\end{displayquote}

It seems that they agree that the app should be more accessible, but that because no current users are reporting accessibility issues nothing will happen. They say that the customer have to request attention to universal design in the project, but earlier in the interview they said that it is up to them, as developers, to focus on accessibility if the customer doesn't have a clear understanding of universal design.

\begin{displayquote}
    The customer wants to get everything out as quickly as possible, so it is us consultants that has to push to make universal design happen.
\end{displayquote}

When the participants were using the Cambridge Glasses, they were asked about how they would feel struggling with not seeing that well everyday. They answered: 
\begin{displayquote}
    \textbf{P} - I would probably not have bothered
    
    \textbf{P} - No. Skipped it all.
    
    \textbf{P} - I would have made someone else do it [laughs].
\end{displayquote}

When they were using Color Carl, a setting in Funkify, one of the participants were asked what they thought:
\begin{displayquote}
    \textbf{I} - What do you think about it?
    
    \textbf{P} - That's too bad.
    
    \textbf{I} - You feel bad for people that has it?
    
    \textbf{P} - Yes. That's how you see colours... They all look sick [referring to people on images on a news website].
\end{displayquote}

\section{Individual}
\subsection{Education}
Jordan and Vanderheiden (2010) says that courses in accessibility and universal design for ICT are only offered at some universities and are almost always electives. Sandnes and Eika (2017) calls this type of course "module based" and it is



\textbf{Can simulations help developers to empathise with users who have different kinds of (dis)abilities?}
All four participants elicitated some sort of response to using simulators. 
