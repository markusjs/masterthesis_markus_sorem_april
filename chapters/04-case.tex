\chapter{Case}
\section{MediaLT}
MediaLT is a Norwegian company concerned with making the web accessible for most people, including people with different kinds of disabilities. They provide training in (among other things) Universal Design and they analyse and comes with suggestions on websites regarding Universal Design principles and implementation.

\section{Stable Test Sites}
This master thesis is related to MediaLT's project "Stable test sites" which came together after Difi in 2016 wanted an overview of how reliable automated testing tools are for measuring WCAG 2.0 success criterion. They found out that the only way to measure this was to find inaccessible websites and see if the automated test tools would react on the errors. They soon found out that this was a tedious job, and that it would be much smarter to make their own inaccessible websites that could measure the WCAG 2.0 success criterion.

This thesis is focused on making Universal Design principles relevant for developers, make it easier to learn about Universal Design principles and to ultimately on their own be able to identify accessibility issues that automatic testing tools fails to identify.

\section{Users}
\textit{Users} in this thesis are front-end programmers, interaction designers or graphic designers, who to some extent has an influence on how an ICT solution aimed at a general population can be designed or implemented. The most relevant users has an ongoing project where they need to cater for universal design in their solution, but as I am also interested in the target users views on disabilities and universal design, all professional designers or front-end developers are target users. 

