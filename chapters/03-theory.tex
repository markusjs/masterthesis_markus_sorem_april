%\begin{displayquote}
    %The power of the web is in its universality. Access by everyone regardless of disability is an essential aspect.
    
    %- Tim Berners Lee
%\end{displayquote}

\chapter{Theory}
\epigraphhead[20]{\epigraph{\textit{The power of the web is in its universality. Access by everyone regardless of disability is an essential aspect.}}{{\textbf{Tim Berners-Lee} \\ Inventor of the World Wide Web}}}

\section{Human-Computer Interaction (HCI) and Interaction Design}
\acrshort{hci} is an academic field concerned with "understanding the influence technology has on how people \textit{think, value, feel and relate}" and how this understanding can inform the design of technology \parencite{wright_empathy_2008}. Interaction Design is the industrial adaptation of HCI research, and is concerned with the practical design of products with the ultimate goal of supporting people (end users) in their everyday and working lives \parencite{rogers_interaction_2011}.

%Both HCI and Interaction Design is concerned with understanding people and how they use ICT solutions and how the user experience can be improved in existing solutions or intact/integrated in future solutions.

One of the earliest mentions of the term HCI was in 1976 in a research paper about office automation \parencite{Carlisle1976}. Carlisle problemize the trend of integrating technology into the workspace without acknowledging how people within the organisation do their work and how human beings function and socialise. 
\begin{displayquote}
    Too many computer-based systems have already been designed on the basis of technological breakthroughs and innovations which were insensitive to the limit on man's rationality and the social needs that must be satisfied within organizational structures.
\end{displayquote}

Carlisle thought the implementation of computers in the workplace was backwards. The focus was on hardware and software, not on how people work or function. Carlisle wanted to use ethnographic methods to study behavior over time, especially structured observation.

\begin{displayquote}
    More recently a research group at the USC/Information Sciences Institute has been using a form of structured observation to study human dialogue as a means of improving man-machine interaction.
\end{displayquote}

It is clear that Carlisle saw the problems with integrating technology that people were going to use without knowing how they normally did their job and communicated with each other (human-to-human).
\section{User Experience}
\subsection{Usability}
\subsection{Accessibility}
\section{Universal Design}
\subsection{Disability Awareness}
\textcite{Jordan2010} says that the current practise in the education programs for engineers and ICT students have been is lacking information about universal design and people with disabilities. As a result, few professionals will have been exposed to the barriers faced by people with disabilities and likely maintain them. 

Disability awareness has the aim of changing attitudes towards people with disabilities. Negative attitudes can be expressed "through avoidance, anxiety, overprotectiveness, pity, segregation, alienation, and rejection" (Elliot \& Byrd 1982 in \cite{Jordan2010}). Jordan request the need for disability awareness in the engineering and ICT programs.

\textcite{Jordan2010} lists four broad approaches to learn about and change attitudes towards people with disabilities: education, facilitated contact with people with disabilities, role playing and disability simulation: 
\begin{itemize}
    \item Education for people without disabilities
    
    Accurate information is used to inform and create understanding on disabilities.
    \item Facilitated contact with people with disabilities
    
    Direct contact with people who experience disabilities. This can challange the students preconceived thoughts on disabilities. 
    \item Role Playing
    
    Students are put to imagine and take the role as a disabled person in order to empathise with some of the experiences they might have.
    \item Disability simulation
    
    Physical or sensorial limitations are used while activities are conducted.
\end{itemize}

\textcite{Jordan2010} argue that education, facilitated contact and role playing require a significant amount of time, but that disability simulations can be quick and effective to create awareness.



\subsection{Self determination theory (SDT)}
\section{Empathy tools}
While there have been many mentions of simulation tools in the litterature \parencite{GoodmanDeane:2007it, Giakoumis2014, Cardoso2012}
%While there have been many mentions of simulation tools in the litterature, I have only 
\section{Simulations}
According to \textcite{Cardoso2012}, in the field of inclusive / Universal Design \textit{simulations} refer to the use of physical restrainers that enables a person to feel the effects different types of capability-losses might have on a person. The aim is to alter the wearers’ experience of their environment to show how everyday products often disregard (and hence disable) a large number of users, due to a lack of consideration of their capacities.

Software simulations where the use of filters and modification of 

One of the earliest documented use of a simulator in design was in the 1950s. A group of industrial designers wore artifical limbs to empathise with war veterans who had amputated their limbs \parencite[3]{Cardoso2012}. In the 1970s, an approach to design called \textit{empathetic modellling} was used in reaserch and design. Empathetic modelling aims to "take designers out of their comfort zone". 
\begin{displayquote}
    Empathic modelling offers designers the opportunity to develop greater insight and understanding, in order to support more effective design outcomes.
    
    We all approach others with our own assumptions and preconceptions until we learn something different or contradictory. This is why empathic modelling is so important; it pushes people to better understand their own values and belief systems, which result in a move towards less ‘projection’ of their own perceptions onto others.
\end{displayquote}
\section{What can be simulated?}

\section{Critique on using simulations}



\section{Related work}
A lot of different studies has been conducted where simulations in some form has been viewed as a way to increase understanding and empathize with people different than the person using the simulator. 

\subsection{Simulation in user-centred design: Helping designers to empathise with atypical users}
\textcite{Cardoso2012} says that designers often applies self-observation teqhniques when they design. This can lead to difficult, frustrating, dysfunctional or even dangerous interactions for a wider variety of people. People experiencing capability-losses, including old and people with temporary or permanent impairments, are likely to be most affected. A user-centered approach to design advocates for direct contact with end-users throughout the design process. 

However, sometimes clients might not think user involvement is a priority, and no allocation is made for it in the budget. Assessing with the use of user simulations has been used in times where user involvement can be hard to accomplish: 
\begin{displayquote}
    In the field of Inclusive/Universal Design, this has consisted of a person (usually able-bodied) wearing physical restrainers to feel the effects of different types of capability-losses, experienced for instance by people with impairments.
\end{displayquote}

%Different impairments can be simulated:

%The aim is to alter the wearers’ experience of their environment to show how everyday products often disregard (and hence disable) a large number of users, due to a lack of consideration of their capacities.

\subsection{Simulating REAL LIVES: Promoting Global Empathy and Interest in Learning Through Simulation Games}
The aim of this study was to investigate if a game that simulates the life of people from different parts of the world could impact students empathy for people of other nation-states, etnisity and linguisticly.

The researchers has found some evidence of positive outcomes of simulations and empathy excercises:
\begin{itemize}
    \item Quin, Rau and Salvendy found that empathy is a dimension of immersion in game narratives. This can mean that games that offers the player to empathise with the characters are more engaging.
    \item Shieh and Cheng found evidence for empathy contributing to greater satisfaction in games.
    \item Yee and Bailenson conducted research on the use of virtual reality simulation and perspective-taking. Young people embodied the character of either an elderly person or a young character. The researchers concluded that those who embodied the elderly character had fewer stereotypical views on elderly and made more situational over dispositional remarks on the persons actions. 
    
    This might indicate that simulations can impact a shift from the medical to the social model of viewing disabilities [my own remarks].
\end{itemize}

REAL LIVES is a game that simulates how it would be to live as a person in another country. The game is text-based and uses real-world data. The game is aimed at 15 year old teenagers and shows how a life can be expected to unfold. 

The hypothesis of the study:
\begin{enumerate}
    \item Students who play the simulation game will exhibit greater global empathy compared with those in an alternate computer-assisted learning activity
    \begin{itemize}
        \item Supported
    \end{itemize}
    \item Students who play the simulation game will show greater interest in future learning about the countries studied compared with students in the alternate computer-assisted learning activity
    \begin{itemize}
        \item Supported
    \end{itemize}
    \item Character identification will be positively associated with global empathy for those who play the simulation game.
    \begin{itemize}
        \item Supported
    \end{itemize}
\end{enumerate}

The results showed that students who played REAL LIVES showed significantly higher levels of global empathy. The students interest for learning more about the lives of people in other nations were also significantly higher. Students showed interest three weeks after being exposed to the simulation excercise. The students were also observed cheering when their character did well and shouting in dismay when negative events occured which might mean that they identified with their character.

