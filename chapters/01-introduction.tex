%\chapter{Introduction}

\chapter{Introduction}
\epigraphhead[20]{\epigraph{\textit{"It means very little to know that a million Chinese are starving unless you know one Chinese who is starving."}}{{\textbf{John Steinbeck}}}}

%\chapter{Background}

\section{The story of Sven}
Sven is from Sweden, but has recently moved to Norway because of work. During his stay he has come across Norwegians from different parts of Norway, all using words that are unfamiliar to him. Most Norwegians can't expect Sven to understand these words, so they intuitively adapt their communication when speaking to him. Sven, on his side, cannot expect Norwegians he meets to speak Swedish to him and he also has to adapt by using words Norwegians understand. This mutual understanding comes from Sven interacting with Norwegians and from Norwegians empathy with Sven as a foreigner. If this was a one-way conversation, Norwegians wouldn't know that Sven had troubles understanding them, and Sven would have a harder time living in Norway.

\section{Background}
%Digital public services has the potential to increase productivity by automating communication between citizens and public services, and between businesses and public services, according to \textcite{Moderniseringsdepartementet2016}.

Information and communication technology (ICT) is integrated in nearly all parts of modern society. 
92\% of Norwegian citizens had access to the Internet in 2011 and 79\% used it daily. In 2015 97\% had access to the Internet and 90\% used it daily. In three years, time spent on the Internet has grown from 112 minutes (in 2013) to 140 minutes (in 2016) per person on average \parencite{ssb_tid_2017}. The use of digital public services has increased by 235\% from 2010 to 2015 \parencite{Moderniseringsdepartementet2016}. 

But there are large differences in age. 44\% of the group between 75 - 79 years report not using the Internet and only 4\% say they have good skills in using the Internet. "Digital natives" (young people grown up using digital devices) can also struggle when an unclear language is used in the digital sphere \parencite[40]{Moderniseringsdepartementet2016}.

Most people are online all the time and expect instant access to information and services. It is important that ICT services are designed and built in a way that does not exclude people with different prerequisitions. Autonomy, equality and inclusion are all important aspects to keep in mind when we are moving towards a more digital society.

Universal design is an attitude and a tool with the intent to include most people in society regardless of functional abilities, age or education. The intent is not to make separate or "special solutions", but to make the same solution accessible and usable by the largest amount of users. 

There has been an increased awareness and interest in Universal Design of ICT in Norway the last couple of years \parencite{begnum_universal_2017}. Awards such as Doga’s (Design and Architecture Norway) “Innovation price for Universal Design” are dealt to ICT solutions who has “made Norway a more including and open society” (Grafill, 2017). Norway has legislated that all new solutions aimed at the general public must be universal designed and all existing solutions must be universal designed by 2021 \parencite{ministry_of_children_and_equality_act_2018}. However, in a report checking the status of universal design of both private and public ICT solutions in Norway, 54\% of public and 49\% of private solutions were measured as being universal designed with scores ranging from 18 to 79 percent \parencite{difi_digitale_2015}.

%\textcite{Cardoso2012} 
%The reason

%Using a user-centered design approach is often the recommended way to meet universal design guidelines (\cite{fuglerud_link_2013 , Keates2014}). But in practise, inclusion of users with different impairments can be hard to accomplish due to availability and budget/time constraints. As a consequense 

%How disability and impairments are viewed by the people making ICT solutions can affect the quality and degree of universal design. Can empathy tools simulating impairments teach empathy? Can the first hand experience of difficulties by using an ICT solution change views developers might have on people with disabilities? 


%Can sympathy, empathy and compassion be used to explicitly state how 

%Can challanging views developers have on universal design and disabilies be a 

%How disability and impairments are viewed by the people making ICT solutions can affect the quality and degree of universal design






%From bank services to building permits, everything is mostly done digital. Norway is the most digitised country in the world \parencite{dagens_naeringsliv_norge_2017}, and it is important to ensure inclusion of all members of society, including people with various forms of disabilities.  


%The fear of breaking the law and to have to remake ICT solutions has also incentivised ICT professionals to focus on Universal Design. However, only 51\% of 300 websites evaluated by Difi in 2015 was within the requirements of the government in Norway.

%\Gls{accessibility} on the web has come a long way in a few decades. It is now punishable by law in Norway to make a publicly accessible website without meeting the criterion for \Gls{UniversalDesign}, and all existing websites must follow these criterion by 2021. But it can be hard to measure if a website is universal designed or not, and to make sure they stay this way when they get updated over time.

%Norway has come a long way legally when it comes to inclusion in the digital sphere. Norway legislated the Anti Discrimination and Accessibility Act in 2008 demanding ICT solutions aimed at the general public to be accessible. Difi (no: Direktoratet for forvaltning og IKT, en: Agency for Public Management and eGovernment) has the role of evaluating existing solutions using the WCAG (Web Content Accessibility Guidelines) 2.0 technical standard.

%\section{Universal Design}
%Many terms for designing and developing ICT solutions that can be used by as many people as possible exists, like "Universal Design", "Universal Usability", "Universal Access", "Design for All" and "Inclusive Design" \parencite{fuglerud_link_2013}. This thesis uses \textbf{universal design} as an umbrella term for all these approaches because The Norwegian Directorate for Children, Youth and Family Affairs (Bufdir) says that these terms are synonymous \parencite{bufdir_universell_2017}. Universal design is also the term used in 

\section{Motivation}
Adherence to requirements and standards is a frequently used approach to universal design \parencite{fuglerud_link_2013}. While this is often a precondition, it does not solve the whole problem. For example: conformance to WCAG 2.0 guidelines can only solve around half of the issues experienced by visually impaired users. Fuglerud suggests that the requirements are used as a part of a human-centered design (UCD) process. This process incorporated in ISO (the International Organization for Standardization) 9241-210:2010 focuses on the use of the system and include the user in the design and development process.

But inclusion of users with different impairments can be hard to accomplish due to availability and budget/time constraints.





%The motivation to conduct this research comes from both a professional and personal level. The professional reason to learn about Universal Design, is that I want to be ready to work as as an interaction designer after I graduate. During the study, I have found out that Universal Design isn’t always focused on early on in projects unless the team has someone who has an understanding and motivation to make sure the team focuses on Universal Design principles.

%PERSONAL MOTIVATION
%My personal motivation to conduct this research comes from an empathy for a group of people who are not always heard. My mother has worked with cultural activity for people with mental and physical disabilities, and has let me participate in her work since I was a child. Aside my studies, I have worked as a social worker with people with autism. With my background I would say that I have empathy for people with different kinds of impairments, and would like to make the world a little better place to live in for this group. 


%\begin{displayquote}
%Disability is not the impairment itself, but rather attitudes and environmental barriers that result in disability.

%- Unicef
%\end{displayquote}

%The Norwegian Equality and Anti-Discrimination Act aims to \textit{"help to dismantle disabling barriers created by society and prevent new ones from being created."} This includes dismantling digital barriers. The legislation views disability as happening when society does not meet the requirements of the individual. This is known as the social model. 

%Contradicting this view is the medical model seeing disability as something within an individual needing treatment in order to be "fixed" \parencite{begnum_2016_views}.




%In Norway, the criteria for an ICT solution to be regarded as universal designed is that it has to follow the WCAG 2.0 technical standard. \textcite{fuglerud_link_2013} states that although guidelines such as WCAG 2.0 is a precondition to achieve an accessible solution, it might not always be enough. A solution might be technically or theoretically within the requirements of universal design, but still be difficult to use. Some research suggests that WCAG 2.0 only covers around half of the issues encountered by visually impaired users \parencite{fuglerud_link_2013}. 

%\subsection{Views on disability}
%The Norwegian Equality and Anti-Discrimination Act aims to "help to dismantle disabling barriers created by society and prevent new ones from being created." This includes dismantling digital barriers. \textcite[]{begnum_2016_views} says that this legislation is based on a modern right view on disabilities which thinks that everyone should have equal rights and not be discriminated against. The legislation is also based on a social adapted model promoting the dismantling of social constructed barriers.

%\textcite{begnum_2016_views} was surprised when she found out that 77\% of the participants (universal design experts) in her study agrees %(amongst other) with the \textbf{charity model} viewing disability as unworthy, a personal tragedy, and that "disabled persons deserve sympathy, support and help". She also notes that this view is common amongst non-disabled people.

%The social adapted model sees contextual and environmental factors as creating the largest barriers for inclusion, while the charity and the closely related medical model focus more on the individuals impairments. The medical and the charity model are according to Langtree (2010 in \cite{begnum_2016_views}) the most common model used by non-disabled people to define and explain disability. 

\subsection{Empowering developers}
The focus of this thesis is frontend developers working on ICT solutions. The reason for this is that it is often up to frontend developers if the ICT solutions are universal designed or not, especially when it comes to semantic code 

Frontend developers of web based ICT solutions in the private sector is the focus of this project. The private sector 

The reason for this is that it is often up to frontend developers if the ICT solutions are universal designed or not, especially when it comes to semantic code. In the evaluation of \textcite{difi_digitale_2015}, code related accessibility errors was found to be most common. Wrong use of heading tags, 

%The legislated requirements used to evaluate ICT solutions can be appear fuzzy \parencite{begnum_universal_2017}, developers typically have little to no training in making ICT solutions user friendly \parencite{Law:ProgrammerFocusedWebsiteAccessibilityEvaluations:2005} and courses aimed at engineers and ICT students in usability and universal design are only available at some universities and are almost always electives \parencite{Jordan2010}.

\textcite{Freire2008} says accessibility has been a very important issue in web development, but that there is a lack of knowledge among developers on which techniques should be used to make websites accessible. They suggests teaching developers how individuals use assistive technologies and show them how a target user struggles with their solutions.

\subsection{Personal motivation}
My personal motivation to conduct this research comes from an empathy for a group of people who are not always heard. My mother has worked with cultural activity for people with mental and physical disabilities, and has let me participate in her work since I was a child. Aside my studies, I have worked as a social worker with people with autism. With my background I would say that I have empathy for people with different kinds of impairments, and would like to make the world a little better place to live in for this group. 

%Langtree (referred to in \textcite{begnum_2016_views}) states that charity 

%\subsection{Why focus on developers}
%As \parencite[]{lundstrom_perceptions_2015} notes, developers tend to take their own decisions when it comes to design

%\section{Goal}
%The ultimate goal for conducting this research is making universal design principles relevant for developers, make it easier to learn about universal design principles and to ultimately on their own be able to identify accessibility issues that automatic testing tools fails to identify. One way to meet this goal is to find a way to lower the threshold for front-end developers to make universal designed ICT solutions. By introducing empathy tools that simulates different types of impairments developers can experience first-hand how an ICT solution might be experienced, to some degree, by people who has disabilities. Can these tools, when used on a prototype or solution the developer has made, have an impact into their views on disability and digital inclusion?

\section{Research Questions}
%The ultimate goal for conducting this research is to understand how empathy tools might impact developers. 

The ultimate goal for this project is to understand how to motivate more developers to assimilate a universal design mindset. My hypothesis is that by letting developers use tools that can simulate functional impairments while using ICT solutions, this can make them realise where the solution is creating digital barriers and how this can be avoided. This knowledge can then be used in future projects.

%is making universal design principles relevant for developers and make it easier to learn about universal design principles so that developers on their own are able to identify accessibility issues that automatic testing tools fails to identify. My hypothesis is that by letting developers use tools that can simulate functional impairments while using ICT solutions, this can make them realise what the digital barriers in the solutions are and how they might be perceived by people who have different kinds of impairments. This again can make it easier to recognise such barriers in other ICT solutions

%The ultimate goal for conducting this research is making universal design principles relevant for developers and make it easier to learn about universal design principles so that developers on their own are able to identify accessibility issues that automatic testing tools fails to identify. My hypothesis is that by letting developers use tools that can simulate functional impairments while using ICT solutions, this can make them realise what the digital barriers in the solutions are and how they might be perceived by people who have different kinds of impairments. This again can make it easier to recognise such barriers in other ICT solutions

%One way to meet this goal is to find a way to lower the threshold for front-end developers to make universal designed ICT solutions. By introducing empathy tools that simulates different types of impairments developers can experience first-hand how an ICT solution might be experienced, to some degree, by people who has disabilities. Can these tools, when used on a prototype or solution the developer has made, have an impact into their views on disability and digital inclusion?

%To lower the threshold of making universal designed ICT solutions, I first have to understand how universal design is dealt with in the development process, and how it should be dealt with. 

%My first research question is:

%\textit{What are the main constraints or barriers hindering developers from making more accessible ICT solutions?}
%\textit{Can empathy tools be a way of empathising}
%\textit{What are the main factors involved in projects that successfully integrates universal design and accessibility?}

\begin{enumerate} 
\item \textit{How to make accessibility issues more apparent for ICT students or professional developers?}
As 
%\item \textit{What are the main factors hindering or promoting developers in making universal designed ICT solutions?}
\item \textit{Can empathy tools motivate the developers to consider the needs of people with different capabilities?}
    \begin{enumerate}
        \item What needs to be present to enable developers to be motivated?
    \end{enumerate}
\end{enumerate} 

 
%My first research question:  


\begin{enumerate} 
%\item \textit{What are the main factors hindering or promoting developers in making universal designed ICT solutions?}
\item \textit{How does the use of empathy tools impact developers?}

\end{enumerate} 
By letting frontend developers use empathy tools on different ICT solutions, I aim at getting a better understanding on the impact this experience have.
%I'm also exploring the use of empathy tools which simulates various impairments and hope to see if this can impact the views front-end developers might have on people with disabilities. 
%I also want to review existing software simulators with developers to see if they can be helpful in their work on developing Universal Designed ICT solutions.

%The technical standards used to evaluate ICT solutions can be hard to fathom, and the most common way used by developers to make sure an ICT-solution is accessible and meeting guidelines and best practises in ICT-projects, is using automatic testing tools that gives feedback on the quality on \textit{code itself} with regards to being technical accessible by for example screen readers.

%Automatic testing tools are valuable in some degrees, for example in making sure all images on a website has an existing alternative text. However, human cognition and usability can not be tested automatically - we then need to think about how the end-user can experience the solution. 

%Empathy is described as understanding the other and seeing a situation from the perspective of others \parencite{lundstrom_perceptions_2015}. My hypothesis is that simulations can provide some low-cost insight into the experience others might have with the artefact. By introducing developers to simulation techniques they can use in their work on ICT-solutions, it can expand their understanding of the diverse abilities and disabilities of people who can become end-users of their solutions and to see the solution from their perspective. Simulations can also make it easier to recognise accessibility issues earlier and learn how to recognise these issues in later projects.

There are many kinds of empathy tools aimed at conveying how different impairment might be experienced. In terms of usability and cost/benefit factors, I analyse which tools might suit frontend developers best.

My second research question is therefore:
\begin{enumerate}\addtocounter{enumi}{1}
\item \textit{Which kinds of empathy tools are best suited for developers working on ICT solutions?}
\end{enumerate} 
%Through a deep dive into existing research on the topic, I hope to explain how simulations can be used in the context of working with ICT-solutions with a frontend developer-centered focus.

%There are many existing simulation techniques and simulation software, and I want to evaluate these with frontend developers. The end goal would be to have requirements of a simulator that is fitting the needs of the developer, and that can help making sure the solution adheres to laws, guidelines and best practises. My third research question is therefore:
%\begin{enumerate}\addtocounter{enumi}{2}
%\item Which simulation techniques or software can be used to make it easy and cost-beneficial for frontend-developers working with ICT-solutions.
%\end{enumerate}
%Simulating impairments means that prototypes can be evaluated by humans, rather than automated testing tools. Many accessibility issues can be found before real users are introduced to the prototype, and the designers and developers can be trained to build ICT solutions more accessible.
%The thesis also has a practical part where I want to build a simulator and evaluate this by including developers and designers and following a User Centered Design (UCD) process.
% where it should be used
% who will be using it
% what are the requirements
% simulating the use of the simulator??